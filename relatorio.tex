\documentclass[12pt, a4paper, margin=3cm]{article}

\usepackage[T1]{fontenc}
\usepackage[utf8]{inputenc}
\usepackage[portuguese]{babel}
\usepackage{authblk}
\usepackage{enumitem}

\setlist[description]{labelindent=1cm}

\title{\textbf{Relatório do 1º projecto de ASA}}
\author{Rodrigo André Moreira Bernardo \\ ist178942}
\affil{Instituto Superior Técnico}

\date{\today}

\begin{document}

\maketitle

\begin{abstract}
<Dizer qual foi a linguagem utilizada e porque>
\end{abstract}

\section{O Problema}

\subsection{Introdução}
\paragraph{}
Paul Erdős, reconhecido matemático do século XX, colaborou com mais de 500
matemáticos na co-autoria de artigos científicos.
O número de Erdős é definido como a distância colaborativa de uma pessoa a Paul
Erdős. Paul Erdős tem número 0, um co-autor de artigos com Paul Erdős tem número
de Erdős 1, um co-autor de artigos com co-autores de Paul Erdős tem número 2 e
assim por adiante [1].

\subsection{Objectivo}
\paragraph{}
Dado um input que identifique Erdős, um conjunto de colaboradores seus e as
colaborações entre si, determinar os números de Erdős de cada um dos
colaboradores. O output deverá indicar o maior valor, \textit{M}, de número de Erdős
identificado, assim como informação do número de pessoas com número de
Erdős \textit{i}, com 1 $\leq$ \textit{i} $\leq$ \textit{M}.

\section{A Solução}
\paragraph{}
A solução passa por executar uma procura em largura primeiro (BFS) sobre um
grafo cujos vértices representam Erdős e os colaboradores, e cujos arcos
representam as colaborações. Com a BFS conseguimos obter a distância
colaborativa entre cada colaborador e Erdős. Por fim, apenas é necessário
efectuar duas passagens pelos colaboradores: uma para determinar \textit{M} e
outra para determinar os valores \textit{i} (ver \textit{Objectivo}).

\subsection{A Representação}
\paragraph{}
Tanto Paul Erdős comos os colaboradores são representados por um inteiro.
A representação do grafo é em listas de adjacências. Mais pormenorizadamente,
este é representado através de um vector de listas ligadas simples, com tamanho
igual ao número de  vértices. Cada lista tem a informação relativa ao vértice
correspondente, utilizada no algoritmo BFS (cor, distância e predecessor), assim
como um apontador para o primeiro elemento da lista. Cada elemento da lista
contém um inteiro representativo do colaborador e um apontador para o próximo
elemento.

\subsection{O Algoritmo}
\paragraph{}
O algoritmo BFS utilizado é um adaptado da página 595 da terceira edição do
livro \textit{Introduction to Algorithms} [2].


\section{Análise Teórica}
\subsection{Avaliação}
\paragraph{}
Do programa destacam-se os seguintes blocos:

\begin{description}
    \item[i.] A inicialização do vector de listas de adjacências, nas linhas
        211-214;
    \item[ii.] A inserção dos arcos no vector, nas linhas 217-233;
    \item[iii.] A BFS, na linha 239;
    \item[iv.] A determinação de \textit{M} (ver \textit{Objectivo}), nas linhas
        242-247;
    \item[v.] A determinação dos valores \textit{i} (ver \textit{Objectivo}), nas linhas
        254-258;
    \item[vi.] A impressão do output, nas linhas 261-265;
    \item[vii.] As operações de alocação e libertação de memória.
\end{description}

Sejam \textit{V} o número de vértices e \textit{E} o número de arcos do grafo.
A complexidade das operações realizadas nos pontos \textbf{i.}, \textbf{iv.},
\textbf{v.} é \textit{O(V)} (ciclos \textit{for} no tamanho do número de vértices).
Em relação ao ponto \textbf{ii.}, deve ser referido que a operação de inserção
em lista é \textit{O(1)}, pois cada novo elemento é inserido no início da lista.
Assim, como a operação \textit{scanf} também é constante, concluí-se que o tempo
de execução deste ponto é \textit{O(E)}, visto que é um ciclo \textit{for} no
tamanho do número de arcos.
O tempo de execução do algoritmo BFS é analisado em [2], na página 597, sendo
este \textit{O(V + E)} (ponto \textbf{iii.}).
Como \textit{M} $\leq$ \textit{E}, sai que a complexidade relativa ao
ponto \textbf{vi.} é \textit{O(E)}.
Por fim, relativamente ao ponto \textbf{vii.}, alocou-se memória para o vector
de listas de adjacências, para as listas em si, para os seus elementos e para a
\textit{queue} utilizada no algoritmo BFS. Tem-se que o número de listas é igual
a \textit{V}, que o número de elementos que têm é \textit{O(E)} e que o tamanho
da \textit{queue} é \textit{O(V)}.

\subsection{Conclusões}
Conclui-se que o tempo de execução do programa é \textit{O(V + E)}.
A memória ocupada é, também, \textit{O(V + E)}.


\section{Avaliação Experimental}
\paragraph{}
<Avaliação Experimental>
Os tempos de execução foram obtidos com a ferramenta \textit{time} do
\textit{UNIX}.

\section{Referências}
\paragraph{}
<Referências>

[1] Enunciado do primeiro projecto de ASA
[2] Introduction to Algorithms

\end{document}
