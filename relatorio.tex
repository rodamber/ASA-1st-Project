Instituto Superior Técnico
[Imagem com o logótipo]
Relatório do primeiro projecto de ASA
Rodrigo André Moreira Bernardo
ist178942

*Resumo
*O Problema
**Introdução
**Objectivo
*A Solução
**A Representação
**O Algoritmo
*Avaliação da Complexidade
*Avaliação Experimental
*Conclusão
*Referências

*Resumo
           (Por fazer)
(Dizer qual foi a linguagem utilizada e porque)


*O Problema
**Introdução

Paul Erdős, reconhecido matemático do século XX, colaborou com mais de 500
matemáticos na co-autoria de artigos científicos.
O número de Erdős é definido como a distância colaborativa de uma pessoa a Paul
Erdős. Paul Erdős tem número 0, um co-autor de artigos com Paul Erdős tem número
de Erdős 1, um co-autor de artigos com co-autores de Paul Erdős tem número 2 e
assim por adiante. [1]

**Objectivo

Dado um input que identifique Erdős, um conjunto de colaboradores seus e as
colaborações entre si, determinar os números de Erdős de cada um dos
colaboradores. O output deverá indicar o maior valor, M, de número de Erdős
identificado, assim como informação relativa ao número de pessoas com número de
Erdős i, com i ∈ [1,M].


*A Solução

A solução passa por executar uma Breadth-First Search (BFS) sobre um grafo cujos
vértices representam Erdős e os colaboradores, e cujos arcos representam as
colaborações. Com a BFS conseguimos obter a distância colaborativa entre cada
colaborador e Erdős. Por fim, apenas é necessário efectuar duas passagens pelos
colaboradores: uma para determinar M e outra para determinar os valores i (ver
Objectivo).

**A Representação

Tanto Paul Erdős comos os colaboradores são representados por um inteiro.

A representação do grafo é em listas de adjacências. Mais pormenorizadamente,
este é representado através de um vector de listas ligadas simples, com tamanho
igual ao número de  vértices. Cada lista tem a informação relativa ao vértice
correspondente, utilizada no algoritmo BFS (cor, distância e predecessor), assim
como um apontador para o primeiro elemento da lista. Cada elemento da lista
contém um inteiro representativo do colaborador e um apontador para o próximo
elemento.

**O Algoritmo

O algoritmo BFS utilizado é um adaptado da página 595 da terceira edição do
livro Introduction to Algorithms [2].


*Avaliação da Complexidade

*Avaliação Experimental

*Conclusão

*Referências

[1] Enunciado do primeiro projecto de ASA
[2] Introduction to Algorithms
